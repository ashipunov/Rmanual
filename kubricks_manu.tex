\documentclass[12pt]{report} % change if you want, e.g., smaller fonts
\usepackage{%
graphicx,
hyperref,
geometry,
paratype} % change the font package if needed

% Page size (A5), change it to the size you need
\geometry{paperheight=210mm,paperwidth=148mm,margin=14mm,top=26mm}

% Define sections
\newcommand{\FF}[1]{\section{#1}}
\renewcommand{\thesection}{\arabic{section}} % there are no chapters

% Define species names
\newcommand{\KK}[1]{\textit{#1}}
\newcommand{\SP}[1]{{\bigskip\noindent\textbf{#1}\par\nopagebreak[4]\medskip}}

% Define dscriptions
\newcommand{\DD}[1]{{\medskip\noindent\leftskip1.2em\small\textbf{Description:} #1.\par}}

% Define images
\newcommand{\I}[1]{\includegraphics{images/#1}\quad}
\newcommand{\II}[1]{{\nopagebreak\medskip\raggedright\noindent\leftskip1.2em #1\par\bigskip}}

% Title page
\title{How to Know Kubricks}
\author{Illustrated Manual}
\date{\today}

\begin{document}

\maketitle

\tableofcontents

\sloppy % useful because it is hard to control "0body" manually

\vspace*{8ex}

In this manual, kubricks are named with scientific binomial names, so
common name of, e.g., \emph{Kubrickus beus} is ``kubrick B''. It is
assumed that all kubricks belong to one genus, therefore largest
subdivisions are \textbf{subgenera}. According to the rules of biological
nomenclature, name of one subgenus (``autonym'') should be identical to
genus name: \emph{Kubrickus}.

\input 0body

\end{document}
